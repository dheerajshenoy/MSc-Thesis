\section{Literature Review}

The motivation for indirect detection of CMEs from Sun has led researchers to find out association of CMEs with eruptive events on the Sun like Solar flare, filament and prominence eruptions, coronal dimming associations with CME and vice-versa etc. About 73 \% association of CME events with eruptive filaments  has been observed \citep{Sinha2019}. It has been demonstrated that EM distributions from SDO/AIA data alone can overestimate the amount of high temperature (logT \textgreater\ 6.4) plasma in the solar corona by a factor of 3-15 \citep{Athiray2024}. Coronal dimming phenomenas have been extensively studied and many empirical formulations have also been derived to get information about the underlying CMEs like mass and velocity of the CMEs ejected by knowing the depth and slope of the dimming curve \citep{Mason2016}. The association of dimming with a CMEs and assocation of CME with dimming has been observed to be very high ($P(Dim \mid CME) = 0.842$, $P(Dim \mid !CME) = 0.167$, $P(CME \mid Dim) = 0.970$) \citep{Veronig2021-rf}. Sun-as-a-star analysis methods have been used widely for comparing and studying stellar and solar CMEs. Multi-temperature structure analysis of stellar active phenomena in spatially integrated spectra is allowed by the combination of H$\alpha$ and EUV lines, instead of single spectra analysis \citep{Otsu2024}. Redshifted components of stellar filament eruptions in Sun-as-a-star analysis in H$\alpha$ spectra may develop into CMEs \citep{Otsu2022}. Further studies into stellar CMEs connecting the CME mass and velocity to the stellar flare energy, leading to the conclusion that stellar CMEs are restricted in terms of their velocity due to the strong stellar magnetic fields and stellar wind drag \citep{Moschou2019-zq}. One expects that the case of stellar CME in terms of it's energetics and association to stellar flares will be a scaled version of the solar CMEs, but discrepancy has been found in terms of the ratio between the Kinetic energy of stellar CME to the stellar flaring energy not being anywhere close to being a scaled version of the solar case \citep{Namekata2022-dm}.

%%% Local Variables:
%%% mode: LaTeX
%%% TeX-master: "main"
%%% End:
