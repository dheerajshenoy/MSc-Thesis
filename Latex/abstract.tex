
\setcounter{page}{1}
\section{Abstract}

%Stellar coronal mass ejections are difficult to observe and analyse due to the lack of spatial resolution. Indirect methods have been devised to detect the signatures of CMEs on stars. We have done Sun-as-a-star analysis of CMEs for various events using the data from SDO/AIA instrument and good correlations have been observed by Differential Emission Measure (DEM) analysis.

%We investigated the temperature variations of solar corona during Coronal Mass Ejections through Differential Emission Measure (DEM) analysis of three CME events. We have looked into the dimming of the corona due to the ejections of plasma, called as coronal dimming, and studied the temperature range of the coronal plasma that's most affected by this dimming. We analysed the filament eruption associated CME ejection and Groundlevel Enhancement CME event and studied their affect on the coronal plasma temperature. Finally, we performed a Sun-as-a-star DEM analysis through pointification of full disk image of Sun to a point source and then look into the temperature variations of the point source. We have found good correlation between the point source and the full disk source in the temperature ranges less than logT = 6.5

Coronal Mass Ejections (CME) are expulsions of plasma from the Sun's corona into the heliosphere due to magnetic reconnection. The study investigates the temperature variations of solar corona during CME events on the Sun through a differential emission measure (DEM) analysis. Leveraging data from Atmospheric Imaging Assembly (AIA) of Solar Dynamics Observatory (SDO), we apply DEM inversion technique to derive the temperature distribution of coronal plasma. We analyse three distinct CME events on the Sun, each associated with unique solar phenomena. The first event is characterized by Coronal Dimming, a phenomena linked to the mass loss from the solar corona. The second event is associated with a large filament eruption and the third event is a ground-level enhancement CME event. In addition to traditional DEM analysis, we investigate which temperature plasma exhibits the most significant dimming during CME events which are associated with coronal dimmings. By correlating the dimming intensity with temperature, we identify the temperature range associated with the most pronounced dimming, providing valuable information on the underlying physical mechanisms driving coronal dimming phenomena. Using an approach of converting full disk image into a point source, we conduct DEM analysis to examine the temperature distribution of the point source and compare it to the full disk image DEM. A good correlation is found in DEM curves of the point source and full disk source through DEM analysis in the temperature range of logT = [5.85, 6.45]. This means that the point source converted Sun shows characteristic signatures showed by full disk Sun during coronal dimming, filament eruption and Groundlevel Enhancement CME below logT $\le$ 6.45.\\

\noindent
\textbf{Keywords}: Solar Physics, Coronal Mass Ejections, Atmospheric Imaging Assembly (AIA), Differential Emission Measure (DEM)

%%% Local Variables:
%%% mode: LaTeX
%%% TeX-master: "main"
%%% End:
