\section{Abstract}

%Stellar coronal mass ejections are difficult to observe and analyse due to the lack of spatial resolution. Indirect methods have been devised to detect the signatures of CMEs on stars. We have done Sun-as-a-star analysis of CMEs for various events using the data from SDO/AIA instrument and good correlations have been observed by Differential Emission Measure (DEM) analysis.

We investigate the temperature variations of solar corona during Coronal Mass Ejections through Differential Emission Measure (DEM) analysis of three CME events. We study the dimming of the corona due to the ejections of plasma, called as coronal dimming, and look into the temperature range of the coronal plasma that's most affected by this dimming. We study a filament eruption associated CME ejection and Groundlevel Enhancement CME event and study their affect on the coronal plasma temperature. Finally, we perform a Sun-as-a-star DEM analysis through pointification of full disk image of Sun to a point source and then look into the temperature variations of the point source.

%%% Local Variables:
%%% mode: LaTeX
%%% TeX-master: "main"
%%% End:
